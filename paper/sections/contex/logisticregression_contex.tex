\subsubsection{Logistic Regression}
\textbf{Definition}
Logistic regression is a statistical model used to predict the probability of a binary class. It is efficient for linear problems and serves as a baseline for comparison with more complex models.

To reference one of the earliest formulations of the model, we draw upon the perspective presented in \textit{The Regression Analysis of Binary Sequences} by D. R. Cox (1958), which introduced a rigorous statistical framework for the analysis of binary outcomes. In this work, both dependent and independent variables are explicitly identified, and the probability of a binary event is modeled using a logistic transformation of the predictors. Dependent variables are represented as binary categorical variables, typically coded as 0 and 1, which enables the analysis of how one or more explanatory variables influence the likelihood of an event of interest. Through illustrative examples, the author demonstrates how the model can be estimated using maximum likelihood methods, thereby establishing a methodological foundation that continues to be widely applied in fields such as biostatistics, the social sciences, and machine learning \citep{cox1958logistic}.

Given the capabilities offered by the model, it has been selected for use in this study. However, prior to its implementation, we will review some previous approaches used in earlier research.

As one of the initial and relevant approaches, the study presented in the article "Offensive Language Detection Using Soft Voting Ensemble Model" \citep{fieri2023offensive} incorporates logistic regression as part of its ensemble strategy. The article emphasizes the use of L2 regularization, which enhances the precision of offensive language classification by mitigating overfitting. Furthermore, the model's simplicity is highlighted as a significant advantage, enabling efficient implementation even with limited computational resources—demonstrating its usefulness and versatility within the set of evaluated classifiers.

The research article titled "Comparison between Machine Learning and Deep Learning Approaches for the Detection of Toxic Comments on Social Networks" \citep{bonetti2023comparison} employs logistic regression once again, combining it with two semantic modeling techniques: Latent Semantic Analysis (LSA) and Latent Dirichlet Allocation (LDA). These methods will be further discussed in the following paragraphs. Notably, the study reports a prediction accuracy exceeding 91\%, highlighting the competitiveness of traditional approaches compared to more complex deep learning models.

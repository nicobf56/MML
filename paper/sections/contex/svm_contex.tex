\subsubsection{SVM (Support Vector Machines)}

SVM (Support Vector Machines) is a machine learning method mainly used for classification tasks. Its main goal is to find the best hyperplane that separates the classes and maximizes the margin with the closest points from each class. This process involves identifying the support vectors, which are the closest points to the classification hyperplane. A common technique used with SVM is the use of kernels. With kernels, data points are transformed into a new space that might be linearly separable, improving classification results. There is a variety of kernels with different transformation approaches to choose from. An important hyperparameter for this algorithm is “C,” which refers to a regularization term that controls the trade-off between margin maximization and misclassifications~\cite{geeksforgeeks_svm}.
